\chapter{最初に}
\section*{まえがき}
このPDFは「なんかいい感じに解析学やってきたのをまとめられないかなあ」みたいなノリでできた事物でして、人が見るのを二の次に考えている代物ですので、まあそれくらいの感じでおおらかにみてください……

一応予備知識として読む前に「これくらいあるといいかな」って想定しているのは高校数学での数学II+B(2021年度での)で登場する微積分法かなって感じです。とはいえ高三になるまであの区分を(駿○かなんかが勝手に決めたのかなあ)なんて勘違いしていた人間が編纂しているので、かなり噛み砕くことにしました。もしかしたら高校数学でのII+Bと被ることもあるかもしれません。ここらへんも生暖かい目で見てやってください……。

最後に、なにか放っておくとまずそうなミスとかいい感じの例とか証明とかあったらTwitterでLex(@\_16777216)へお願いします。
\section*{本文中で使用される記号}

\subsection*{$\mathbb{N},\,\mathbb{Z},\,\mathbb{Q},\,\mathbb{R},\,\mathbb{C}$}
\textgt{それぞれ自然数・整数・有理数・実数・複素数体のことを示す。また、自然数に0は含めないことにし、非負整数は$\mathbb{Z}_{\geq 0}$を用いることにする。}

\subsection*{$\lim$}
\textgt{極限をとる。}

\subsection*{$\{a_n\},\,\{b_n\},\,\{c_n\},\,\cdots$}
\textgt{$n$を項数とする数列を示す。}