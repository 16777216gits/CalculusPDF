\documentclass[book,paper=a5,fontsize=11.75Q,jafontscale=0.91,gutter=12mm,head_space=14.4mm,foot_space=20.736mm,baselineskip=1.625zw]{jlreq}
\usepackage[intlimits]{amsmath}
\usepackage{mathtools,amssymb,amsthm}
\usepackage{newpxtext,newpxmath}
\usepackage{autobreak}
\usepackage{bm}
\usepackage{luatexja}
\usepackage{mleftright}
\usepackage{empheq}
\usepackage{tcolorbox}
\usepackage{graphicx,xcolor,tikz}
\usepackage{makeidx}
\everymath{\displaystyle}
\allowdisplaybreaks
\newcommand{\opdiff}[1]{\frac{\mathrm{d}}{\mathrm{d}#1}}
\newcommand{\diff}[2]{\frac{\mathrm{d}#1}{\mathrm{d}#2}}
\newcommand{\diffn}[3]{\frac{\mathrm{d}^{#3}#1}{\mathrm{d}#2^{#3}}}
\newcommand{\oppdiff}[1]{\frac{\partial}{\partial #1}}
\newcommand{\pdiff}[2]{\frac{\partial #1}{\partial #2}}
\newcommand{\pdiffn}[3]{\frac{\partial ^{#3}#1}{\partial #2^{#3}}}
\newcommand{\dd}{\mathrm{d}}
\newcommand{\df}{\mathrm{d}f}
\newcommand{\dg}{\mathrm{d}g}
\newcommand{\dx}{\mathrm{d}x}
\newcommand{\dy}{\mathrm{d}y}
\newcommand{\dz}{\mathrm{d}z}
\newcommand{\dr}{\mathrm{d}r}
\newcommand{\dt}{\mathrm{d}t}
\newcommand{\du}{\mathrm{d}u}
\newcommand{\dv}{\mathrm{d}v}
\newcommand{\dS}{\mathrm{d}S}
\newcommand{\dV}{\mathrm{d}V}
\newcommand{\N}{\mathbb{N}}
\newcommand{\Z}{\mathbb{Z}}
\newcommand{\Q}{\mathbb{Q}}
\newcommand{\R}{\mathbb{R}}
\newcommand{\C}{\mathbb{C}}
\newcommand{\e}{\mathrm{e}}
\newcommand{\dth}{\mathrm{d}\theta}
\newcommand{\dta}{\mathrm{d}\tau}
\newcommand{\dph}{\mathrm{d}\varphi}
\newcommand{\pare}[1]{\mleft(#1\mright)}
\newcommand{\abso}[1]{\biggm| #1 \biggm|}
\newcommand{\equ}[1]{\begin{equation}#1\end{equation}}
\newcommand{\const}{\mathrm{const.}}
\newcommand{\tsum}[3]{\sum_{#1=#2}^{#3}}
\newcommand{\mat}[1]{\begin{pmatrix}#1\end{pmatrix}}
\newcommand{\vmat}[1]{\begin{vmatrix}#1\end{vmatrix}}
\newcommand{\case}[1]{\begin{empheq}[left=\empheqlbrace]{align*}#1\end{empheq}}
\newcommand{\fm}[1]{\begin{align*}#1\end{align*}}
\newcommand{\dfp}[3]{#1\frac{\mathrm{d}#2}{\mathrm{d}#3}+#2\frac{\mathrm{d}#1}{\mathrm{d}#3}}
\newcommand{\opdfp}[3]{\left(\frac{\mathrm{d}}{\mathrm{d}#3}#1\right)#2+\left(\frac{\mathrm{d}}{\mathrm{d}#3}#2\right)#1}
\newcommand{\cd}{\cdots}
\newcommand{\vd}{\vdots}
\newcommand{\ddo}{\ddots}
\newcommand{\hd}[1]{\hdotsfor{#1}}
\newcommand{\arsinh}{\mathrm{arsinh}}
\newcommand{\arcosh}{\mathrm{arcosh}}
\newcommand{\artanh}{\mathrm{artanh}}
\newcommand{\rot}{\mathrm{rot}}
\newcommand{\grad}{\mathrm{grad}}
\newcommand{\diver}{\mathrm{div}}
\newcommand{\vnabla}{\bm{\nabla}}
\newcommand{\Grad}[1]{\bm{\nabla}#1}
\newcommand{\Div}[1]{\bm{\nabla}\cdot\bm{#1}}
\newcommand{\Rot}[1]{\bm{\nabla}\times\bm{#1}}
\renewcommand{\thefootnote}{*\arabic{footnote}}
\makeindex
\begin{document}

\frontmatter
\chapter{最初に}
\section*{まえがき}
このPDFは「なんかいい感じに解析学やってきたのをまとめられないかなあ」みたいなノリでできた事物でして、人が見るのを二の次に考えている代物ですので、まあそれくらいの感じでおおらかにみてください……

一応予備知識として読む前に「これくらいあるといいかな」って想定しているのは高校数学での数学II+B(2021年度での)で登場する微積分法かなって感じです。とはいえ高三になるまであの区分を(駿○かなんかが勝手に決めたのかなあ)なんて勘違いしていた人間が編纂しているので、かなり噛み砕くことにしました。もしかしたら高校数学でのII+Bと被ることもあるかもしれません。ここらへんも生暖かい目で見てやってください……。

最後に、なにか放っておくとまずそうなミスとかいい感じの例とか証明とかあったらTwitterでLex(@\_16777216)へお願いします。
\section*{本文中で使用される記号}

\subsection*{$\mathbb{N},\,\mathbb{Z},\,\mathbb{Q},\,\mathbb{R},\,\mathbb{C}$}
\textgt{それぞれ自然数・整数・有理数・実数・複素数体のことを示す。また、自然数に0は含めないことにし、非負整数は$\mathbb{Z}_{\geq 0}$を用いることにする。}

\subsection*{$\lim$}
\textgt{極限をとる。}

\subsection*{$\{a_n\},\,\{b_n\},\,\{c_n\},\,\cdots$}
\textgt{$n$を項数とする数列を示す。}

\tableofcontents

\mainmatter
\chapter{一変数関数の実解析}
\section{高校の復習}
高校数学において、いわゆる数学II+Bの学習内容で微積分は最初に登場する。そこでの微分法・積分法の取り扱いの復習やそれらの応用(対数微分法や合成関数の微分法、置換積分法や部分積分法など)を、この文書における解析学への肩慣らしとしてまずは復習の形で取り扱うことにする。
\subsection{微分法}
ある定義域で定義されている任意の一変数関数$f\pare{x}$を考え、これの定義域内の二点$a,\,b$をとることにする。

ここで、二点間での$f$の値の差分を取り、二点間の$x$座標の差分で割ることを考える。式にするなら
\begin{align*}
    \frac{f\pare{b}-f\pare{a}}{b-a}
\end{align*}
となろう。ここで$b:=a+h$とするなら
\begin{align}
    \frac{f\pare{a+h}-f\pare{a}}{h}\label{eq:1.1}
\end{align}
となる。

ここでこの$h$をどんどん小さくしていく($a$と$b$を近づける)ことを考えると、ただ単に$h$に0を外挿するだけでは(\ref{eq:1.1})はゼロ除算となり定義できなくなるがここで極限をとることを考えると
\begin{align}
    \lim_{h\to 0}\frac{f\pare{a+h}-f\pare{a}}{h}\label{eq:1.2}
\end{align}
となる。もしこの極限が存在するとき、この値を$\diff{f}{x}$とおく。これを$f$の($a$における)微分係数という。
とある区間$I$をおいて、その区間内の全ての点で関数$f$が微分可能ならば、$f$は$I$において微分可能といえる。微分係数を求める行為を微分といい、微分して出てきた関数を($f$の)導関数という。

導関数を求めるという一連の行為はかいつまんで言うならば「$a$近傍で$f$ってどんな振舞いをしているのかな」という局所的な部分に注目した行為とも見てとれ、このようなものを局所的概念という。ちなみに、$\diff{f}x$は$f'\pare{x},\,f^{\pare{1}}\pare{x}$などとも書かれるが、この文書では原則、$\diff{f}x$と書くことにする。導関数に値を外挿するとき、$\diff{f}x\Big|_{x=c}$などのように書くこともある。
\subsection{主要な関数の導関数}
\begin{align}
    \begin{aligned}
        \diff{\pare{x^n}}{x}          & =nx^{n-1}     \\
        \diff{\pare{\sin x}}{x}       & =\cos x       \\
        \diff{\pare{\cos x}}{x}       & =-\sin x      \\
        \diff{\pare{\mathrm{e}^x}}{x} & =\mathrm{e}^x \\
        \diff{\pare{\ln x}}{x}        & =\frac{1}x
    \end{aligned}\label{eq:1.3}
\end{align}
高校数学における指数法則などを既知として、また、$\lim_{x\to 0}\frac{\sin x}{x}=1$も既知とすることとしてこれらを証明していく。指数法則は本来$\mathbb{Q}$上から$\mathbb{R}$上へ拡張をせねばならないが、これも天下り的に拡張可能なものとして取り扱う。また、後ろ二本は無条件に使えるものとする(ゆえに、後でもう一度示してやらねばならない)。ここまででわかるように高校数学でのこれらの取り扱いはいくぶん不完全であるので、後でもう一度(指数関数や対数関数などを再定義する形で)触れることにする。

\begin{align*}
    \diff{\pare{x^n}}{x} & =\lim_{h\to 0}\frac{\pare{x+h}^{n}-x^n}{h}              \\
                         & =\lim_{h\to 0}\frac{x^n-x^n+\pare{nhx^{n-1}+\cdots}}{h} \\
                         & =nx^{n-1}.
\end{align*}
括弧内の項は$h$について二次以上($h^2$以上を項にふくむ)なので分母の$h$を約してもまだ$h$が残る。そのため括弧内は第一項を除いてすべて極限をとった際にゼロになり、所望の結果を得る。

\begin{align*}
    \diff{\pare{\sin x}}{x} & =\lim_{h\to 0}\frac{\sin\pare{x+h}-\sin{x}}{h}                \\
                            & =\lim_{h\to 0}\frac{\sin x\cos h+\cos x\sin h-\sin x}{h}      \\
                            & =\lim_{h\to 0}\sin x\frac{\cos h-1}{h}+\cos x\frac{\sin h}{h} \\
                            & =\cos x
\end{align*}
加法定理で展開。$\cos x$の微分も同様な展開を施せばいいのでこちらは略することにする。
\subsection{合成関数微分}
一変数関数$f$において、もし$f$の中身に別の関数$g$がある、というふうな解釈ができるとき、これの微分は
\begin{align}
    \diff{f}x & =\diff{f}{g}\diff{g}{x}\label{eq:1.4}
\end{align}
となる。みため、ただの分数の約分のように扱え、便利である。
\vskip0.75\baselineskip
例:$f\defy\mathrm{e}^{-x^2}$

$g\defy-x^2$とおくと、
\begin{align*}
    \diff{f}{x} & =\diff{f}{g}\diff{g}{x}           \\
                & =\mathrm{e}^{-x^2}\cdot\pare{-2x} \\
                & =-2x\mathrm{e}^{-x^2}.
\end{align*}
\subsection{対数微分法}
対数微分法も捉え方によっては合成関数微分である(ので、あまり別個に名前をつける必要はないんじゃないかな……って思っている)。

\begin{align*}
    \opdiff{x}\pare{\ln f} & =\frac{\df/\dx}{f}
\end{align*}
より、これを変形し
\begin{align}
    f\opdiff{x}\pare{\ln f} & =\diff{f}{x}
\end{align}
ともいえる。

\subsection{不定積分・定積分}
高校数学においては微分積分学の基本定理を自明なものとして、ある関数$F$の導関数が$f$のとき、$F$は$f$の原始関数といえ、原始関数を求める行為を不定積分とみなしていた。
\begin{align}
    \begin{aligned}
        \diff{F}{x}                 & =f        \\
        \Leftrightarrow \int f\,\dx & =F+\const
    \end{aligned}\label{eq:1.5}
\end{align}
ここで$\const$は任意定数。このように原始関数には定数項に任意性が存在する。

定積分は不定積分により求まった原始関数の上下端を指定し、実際にその値を求めることとして取り扱った。
\begin{align}
    \begin{aligned}
        \int_a^b f\,\dx & =[F]_a^b             \\
                        & =F\pare{b}-F\pare{a}
    \end{aligned}\label{eq:1.6}
\end{align}
要するに「積分は微分の逆」として扱ったわけである。
\subsection{置換積分・部分積分}
被積分関数に合成関数微分のような形を適用させることで置換積分法ができる。
\begin{align}
    \int f\,\dx & =\int f\,\diff{x}u\du \label{eq:1.7}
\end{align}
これもまた、見た目上ただの分数の約分のように扱えて明快である。

また、積の微分法から部分積分法が導ける。$f,\,g$を$x$の関数とし$\diff{G}x:=g$とすると
\begin{align}
    \begin{aligned}
        \diff{\pare{fG}}x                          & =\diff{f}xG+f\diff{G}x                 \\
        \Leftrightarrow \int\diff{\pare{fG}}x\,\dx & =\int\pare{\diff{f}xG+f\diff{G}x}\,\dx \\
        \Leftrightarrow fG-\int f\diff{G}x\,\dx    & =\int\diff{f}xG\,\dx                   \\
        \Leftrightarrow \int fg\,\dx               & =fG-\int\diff{f}xG\,\dx
    \end{aligned}\label{eq:1.8}
\end{align}
\clearpage
\section{各種初等関数について}
ここでは初等関数をおさらいして、必要ならば定義を行う。一番最初の積分に関してはこのあと、実数体の完備性を基に裏付けをとるものとして、それ以降の論理での整合性をここで完全にする、という二段階の方式をとることにする。

そのため最初の積分の式のみが高校流の(直感的ながらもふんわりとした)定義にもとづくが、後で直感に反しないようにより厳密に記述を加える。
\subsection{指数関数・対数関数}
これを定義として始めることにする。
\begin{align}
    \begin{aligned}
        \int_1^x \frac{\du}u & \defy\ln x \\
        \ln 1                & =0
    \end{aligned}\label{eq:1.9}
\end{align}

この積分は\textgt{$\mathbb{R}$上全体で定義されている}(厳密にはここで実数体の稠密性を考えてやる必要があるが、これはこのあとにする)。そのため、右辺の関数ももちろん$\mathbb{R}$上全体で定義されている。

さて、ここから指数関数と指数法則その他諸々を「再現」していくことにする。
\begin{align*}
    \ln\pare{xy} & =\int_1^{xy}\frac{\du}{u}                                                                 \\
                 & =\int_1^{x}\frac{\du}{u}+\int_x^{xy}\frac{\du}{u}                                         \\
                 & =\ln x+\int_x^{xy}\frac{\du}{u}                                                           \\
                 & =\ln x+\int_1^y \frac{\dt}u\diff{u}t              & \pare{\text{$u\defy\frac{t}x$とする}} \\
                 & =\ln x+\int_1^y\frac{\dt}t                                                                \\
                 & =\ln x+\ln y                                      & \pare{\text{対数の法則}}
\end{align*}
\begin{align*}
    -\ln{x} & =\ln 1-\ln x         \\
            & =\ln\pare{\frac{1}x}
\end{align*}
対数の和が示せた。

続いて、指数関数$\exp x$を対数関数の逆関数として定義する。
\begin{align*}
    \ln^{-1}x:=\exp x
\end{align*}

先程示した対数の和より、
\begin{align}
    \exp x\exp y & =\exp \pare{x+y}
\end{align}\label{eq:1.10}
となる。ここで、$\exp1:=\mathrm{e}$なる数を定義すると、この式は$x\in\mathbb{Q}$のとき$\mathrm{e}$を底とする指数と同じくなる。そこで、$\mathrm{e}^x$を$\mathbb{R}$上に、この式を用いることで再定義することにする。こうしてここに指数関数は$\mathbb{Q}$上から$\mathbb{R}$上に拡張された。
\subsection{三角関数・逆三角関数}
三角関数に関してはこれまでの一般角を用いた定義で十分だろう。ここで、三角関数に関してその逆関数を考える。ただし一価にならねばならないので、$\sin x,\,\tan x$に関しては$x\in \Big[-\frac{\pi}2,\,\frac{\pi}2\Big]$、$\cos x$に関しては$x\in\pare{0,\,\pi}$として考える。この関数を$\arcsin x,\,\arccos x,\,\arctan x$とする。

$\arcsin x,\,\arccos x$は$[-1,1]$で、$\arctan x$は$\pare{-\infty,\,\infty}$で定義されている。

各々の一階導関数は、
\begin{align}
    \begin{aligned}
        \diff{y}{x} & =\opdiff{x}\arcsin x        \\
                    & =\frac{1}{\dx/\dy}          \\
                    & =\frac{1}{\cos y}           \\
                    & =\frac{1}{\sqrt{1-\sin^2y}} \\
                    & =\frac{1}{\sqrt{1-x^2}}
    \end{aligned}
\end{align}
同様に逆関数の微分を挟むことで$\arccos x,\,\arctan x$の導関数も求まり
\begin{align}
    \opdiff{x}\arccos x & =-\frac{1}{\sqrt{1-x^2}} \\
    \opdiff{x}\arctan x & =\frac{1}{1+x^2}
\end{align}
となる。
\subsection{双曲線関数・逆双曲線関数}
指数関数を用いて定義される双曲線関数を定める。これに関しても逆関数を逆双曲線関数$\pare{\arcosh,\,\arsinh,\,\artanh}$という。
\begin{align}
    \frac{\mathrm{e}^x+\mathrm{e}^{-x}}{2} & \defy\cosh x \\
    \frac{\mathrm{e}^x-\mathrm{e}^{-x}}{2} & \defy\sinh x \\
    \frac{\sinh x}{\cosh x}                & \defy\tanh x
\end{align}
余談だが、arc-ではなくar-である。

指数関数の微分と合成関数微分を用いて双曲線関数の導関数は簡単に求まり、
\begin{align}
    \begin{aligned}
        \opdiff{x}\sinh x & =\cosh x            \\
        \opdiff{x}\cosh x & =\sinh x            \\
        \opdiff{x}\tanh x & =\frac{1}{\cosh^2x}
    \end{aligned}
\end{align}
\vskip1.5\baselineskip

さて、ここまでのもの(三角関数・指数関数とその逆関数)に多項式関数を含めたものをまとめて初等関数と呼ぶ。初等関数でない関数(後述するΓ関数やΒ関数など)を特殊関数と呼ぶ。

積の微分法$\opdiff{x}\pare{fg}=\diff{f}xg+f\diff{g}x\,\,\,\pare{\text{一般に}\opdiff{x}\pare{\prod_{i=1}^n\,f_i}=\sum_{i=1}^{n}\frac{\dd f_i}{\dx}\frac{1}{f_i}\prod_{j=1}^n\,f_j}$や商の微分法$\opdiff{x}\pare{\frac{f}g}=\frac{\pare{\df/\dx}g-f\pare{\dg/\dx}}{g^2}$がある以上「初等関数の微分は初等関数で記述できる」が、積分計算は線形性$\int\pare{f+g}\,\dx=\int\!f\,\dx+\int\!g\,\dx$こそあるものの一般的な積や商に関する公式は存在せず、「初等関数の積分が初等関数で表されるとは限らない」ことは書き記しておくべきことであろう(そして、初等関数の定積分で定義された特殊関数も存在する)。

\clearpage
\section{ε-δ論法}
これまで極限やそれのふるまいを考えるときは高校のときに使った考えをそのまま流用していた。そのときはどう学んでいたか。$\{a_n\}$を考えたとき、$\lim_{n\to\infty}a_n$とは「$n$を限りなく大きくしたときの数列のふるまい(収束するかしないかなど)」として考えていた。関数の極限も$\lim_{x\to c}f\pare{x}$を「$x$を限りなく$c$に近づけたときの$f$のふるまい」として考えていた。

しかしながらこの「限りなく」という語が厄介である。曖昧なこの語を言い換えるとどういうことになろう、と考えると「無限に小さくなるように近づけ……」あたりであろう。これはよろしくない。無限を絡めたものの定義に無限という言葉が出てきてしまう。あまりうまい方法ではない。ほかに「これまでに身につけた、無限という語を介さない表現で無限を表現する方法」はないだろうか\footnote{「別に直感で分かるからいいじゃないか」という考えを持つように感じることもあるだろう。実際、18世紀ほどまではここを掘り下げることのないまま、直感的なまま微分積分学の各種性質や研究がされてきていた。19世紀になって、主にCauchyなどにより定式化される形でε-δ論法は成立した。}。
\skp{1}

そこで考えられたのがε-δ論法である。要は、無限に近づくみたいな言い回しを避けたいので、ここで「求めたい極限$c$と$a_n$の距離の差」に対応する「誤差」$a_n-c$がある実数$\varepsilon$以内で許容されるとする。もし$\{a_n\}$が$c$に「限りなく近づく」のなら、ある値以上の$a_n$について誤差が必ず$\varepsilon$以内に収まるであろう。こうすれば有限の値のみを用いて「限りなく」が表現できる。このとき、$\varepsilon$以内に入らないギリギリのものを$a_\delta$とする。

つまり、これまでの言い方で「$a_n$が$c$に限りなく近づく」といえる数列$\{a_n\}$があったとき、それはつまり任意の$\varepsilon$を用いることで「$\delta<n$の$a_n$については必ず$c$から$\varepsilon$のところにある」ということがいえる(逆に、それ以外の$n$については何もわからなくてよい)。

\begin{defi}
    任意の$\varepsilon\!>\!0$に対し「$\delta$より大きい$n$について$|a_n-c|\!<\!\varepsilon$が成立する」$\delta$が存在することを、$\lim_{n\to\infty}a_n=c$ということにする。
\end{defi}

ここに数列の極限が厳密に定まった。関数の極限も同じように考えて定義しなおそう。感覚的には同じように$\lim_{x\to{}a}f\pare{x}=b$のことを「誤差として$\varepsilon$を許容したとき、$b$から$\varepsilon$の範囲内に$f\pare{a-\delta}$から$f\pare{a+\delta}$までの全ての$f\pare{x}$が入る」ことを考えればいいことになる。

\begin{defi}
    区間$I$と点$a\!\in\!I$が与えられたとき、$I\,\backslash\,a$で定義された関数$f\pare{x}$があるとする。

    ここで任意の$\varepsilon\!>\!0$に対し「$|x\!-\!a|\!<\!\delta$をみたすような$I$上の全ての点に対し、$|f\pare{x}-b|\!<\!\varepsilon$が成立する」$\delta$が存在することを、$\lim_{x\to{}a}f=b$ということにする。
\end{defi}




\clearpage
\section{実数体の完備性}
この節ではこれまで暗黙に認めていた(そして、直感に反しないので「まあそうであろう」と感じてもいた)実数体の完備性について記述する。

$\mathbb{N},\,\mathbb{Z},\,\mathbb{Q}$は(厳密性を欠いて直感的に言えば)「すき間」が存在する。数直線上のどこを探しても$\sqrt{2}$を表す点は$\mathbb{Q}$にも$\mathbb{Z}$にも、もちろん$\mathbb{N}$にも存在しない。だが$\mathbb{R}$上にはどんな点でも存在する。これを厳密に公理として採用する(これについては証明無しに認めて、ここから定理を導く、ということになる)。とはいえ、ここではあくまで「そういう前提がある」紹介にとどめ、実数体の完備性及び有界単調数列の収束性から導ける、主要でない各種定理の証明はある程度補遺にまわすことにする\footnote{以降、補遺にこのような形式的な証明をまわすことが多々あると思われるが、これは数学的道具として解析学をみている人間と厳密な証明を追いたい人間の大きく分けて二通りがいることなどを考慮するとこれがいいかと勘案した結果のもの。} 。

\appendix
\chapter{定積分の数値的計算}

\printindex

\end{document}